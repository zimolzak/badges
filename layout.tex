\documentclass[12pt]{article}
\usepackage{badges} %use xelatex
\begin{document}


\begin{tabular}{|c|c|} % p{4in}

\hline
\rule{0pt}{1.5in}First & First \\
Last & Last \\
\rule{0pt}{1.5in} & \\
\hline

\rule{0pt}{1.5in}First & First \\
Last & Last \\
\rule{0pt}{1.5in} & \\
\hline

\rule{0pt}{1.5in}First & First \\
Last & Last \\
\rule{0pt}{1.5in} & \\
\hline

\end{tabular}

Children’s Hospital Informatics Program. Analyze large pharmacy claims
databases to develop predictive models of medication adherence. Use
SAS software to operate on the primary data set from Aetna, Inc.,
which selected commercial insurance members with hyperlipidemia and
comprised 3 major database tables with 61 million enrollment records,
200 million medical claims records, and 90 million prescription claims
records. Secondary data sets included laboratory and survey data from
Aetna, and data from a different insurer on different chronic
conditions. Perform modeling using techniques of multivariable
logistic regression and multivariable adaptive regression splines.
Quantify prescription refill irregularity using Fourier spectral
analysis. Present at and attend regular laboratory meetings and
journal clubs. Coursework including Harvard School of Public Health
Program in Clinical Effectiveness (introductory epidemiology,
biostatistics, health services research, pharmacoepidemiology); and
coursework in biomedical informatics, data mining statistical machine
learning, population health informatics, business management, and
grant writing. Attend American Medical Informatics Association
symposia and National Library of Medicine informatics conferences.
Professor: Kenneth Mandl, Harvard Medical School. Funded by National
Library of Medicine under grant number 5-T15-LM007092-20, to Boston
Informatics Training Program. Program director: Alexa McCray, Harvard
Medical School.

\end{document}

